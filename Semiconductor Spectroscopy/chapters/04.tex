\chapter{Evaluation}

\section{Spectra of Different Light Sources}

The light source used for this experiment is a glowing light bulb. It is chosen due to its continuous spectrum in the desired wavelength range. Examples for less continuous spectra would include fluorescent light tubes, LEDs or LASERs. The first two are examined before the actual experiment for comparison. An LED flashlight and a fluorescent light tube desk lamp are used as sources. 

As can be seen in Fig. \ref{fig:ss_ex0}, the spectrum of the LED shows one continous Lorentz-like distribution from $\lambda \approx 400$ to \SI{800}{nm} and one narrower peak at $\lambda \approx \SI{450}{nm}$. The latter is the actual characteristic transition in the LED chip itself, while the rest of the spectrum is caused by absorption of the blue wavelength photons and re-emission via the Stokes shift. 

Fig.\ref{fig:ss_ex0} depicts the spectrum of a fluorescent light tube. It consists of a multitude of peaks, though all of them are sharp and localized. They are caused by discrete atomic transitions. The reasons for their width not being exactly zero include the Doppler-effect, energy-time uncertainty, external fields and atomic collisions. 

\begin{figure}[h]
    \centering
    \includegraphics[width=0.8\linewidth]{Semiconductor Spectroscopy/ss_ex0.jpg}
    \caption{Spectra of different light sources.}
    \label{fig:ss_ex0}
\end{figure}



\section{Exercise 1: Measurement of Colloid Radius}
For this exercise, the size of the colloids in the CdS/CdSe glass rod sample is determined by means of absorption edge shift. Depending on the colloid size, the absorption energy and wavelength are shifted. This shift can be described by eq. \ref{dot_energy}. By rearranging the formula, the minimum colloid radius is given by

\begin{equation}
    r_\mathrm{min} = \sqrt{\frac{h}{8 m c (\frac{1}{\lambda_\mathrm{blue}} - \frac{1}{\lambda_\mathrm{red}})}}
\end{equation}

This yields TODO

\section{Exercise 2: Measurement of Layer Thickness}
In this exercise, the layer thickness of the CdS sample is analyzed. To do so, a polarizer is used between the sample and detector. It was also used for the reference spectrum. Two spectra with a relative polarization angle of 90° were recorded. 
