\chapter{Measurement Procedure}
\section{Colloid Sample (CdS/CdSe Glass Rod)}

Transmission spectra are measured at multiple positions along the rod (400--\SI{800}{nm}). The absorption edge shifts as colloid size varies along the gradient, enabling extraction of the minimal colloid radius.

\section{CdS Crystal}

Room-temperature transmission spectra are recorded (450--\SI{800}{nm}) for two orthogonal polarizations (\(\boldsymbol{E} \parallel \boldsymbol{c}\) and \(\boldsymbol{E} \perp \boldsymbol{c}\)), resulting in spectra of maximum and minimum transmission. Fabry-Pérot fringes visible in the spectra can be used to determine the layer thickness.

\section{GaAs/AlGaAs Multiple Quantum Well}

The MQW sample is cooled to 77~K by liquid nitrogen filling. Transmission spectra (600--820~nm) are acquired after thermal stabilization. The reduced temperature sharply resolves the \(n = 1\) heavy-hole and light-hole exciton doublet. The energy splitting of these excitons yield the well thickness.

\section{Cu\(_2\)O Crystals }

Absorption spectra are recorded for samples with different thicknesses, where the np exciton series can be identified. From these, the band-gap energy and binding energy are to be determined, as well as the binding energy of the 1s exciton.