\chapter{Measurement Procedure}
\section{Colloid Sample (CdS/CdSe Glass Rod)}

Transmission spectra are measured at multiple positions along the rod (400--\SI{800}{nm}). The absorption edge shifts as colloid size varies along the gradient, enabling extraction of quantum confinement effects and band-gap energy as a function of quantum dot radius.

\section{CdS Crystal}

Room-temperature absorption spectra are recorded (450--\SI{800}{nm}) for two orthogonal polarizations (\(\boldsymbol{E} \parallel \boldsymbol{c}\) and \(\boldsymbol{E} \perp \boldsymbol{c}\)). The uniaxial crystal exhibits dichroism: A-excitons appear only for \(\boldsymbol{E} \perp \boldsymbol{c}\), while B-excitons are visible for both polarizations. Fabry-Pérot fringes determine layer thickness; thermal broadening at room temperature typically resolves only the \(n_B = 1\) exciton.

\section{GaAs/AlGaAs Multiple Quantum Well}

The MQW sample is cooled to 77~K by liquid nitrogen filling. Transmission spectra (600--820~nm) are acquired after thermal stabilization. The reduced temperature sharply resolves the \(n_z = 1\) heavy-hole and light-hole exciton doublet. The energy splitting and spectral positions yield well thickness.

\section{Cu\(_2\)O Crystals (Two Thicknesses)}

Absorption spectra are recorded for samples with different thicknesses. Cu\(_2\)O exhibits dipole-forbidden 1s exciton transitions. The absorption spectrum displays the LO-phonon-assisted square-root-shaped background with superimposed \(n_B p\) exciton series (\(l = 1\) states) for \(n_B = 2, 3, 4, \ldots\). Comparison between thin and thick samples isolates the intrinsic absorption, enabling determination of the 1s exciton binding energy.