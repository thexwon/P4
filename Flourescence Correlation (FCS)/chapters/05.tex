\chapter{Summary}

In this experiment, Fluorescence Correlation Spectroscopy (FCS) was performed using a confocal microscope to study fluorescent particles in the nanomolar range. The setup was first characterized by estimating the lateral resolution of the microscope, which was found to be approximately \(0.363 \pm 0.006 \, \mu\text{m}\) based on immobilized fluorescent beads, consistent with a rough estimate from imaging an EPROM chip.

The FCS setup was calibrated using an Atto655 reference sample, yielding an effective focal volume of \(0.56 \pm 0.05 \, \text{fL}\). Subsequent FCS measurements on six nanoparticle samples allowed determination of their hydrodynamic radii and concentrations via autocorrelation analysis and the Stokes–Einstein relation. The nanoparticles exhibited similar sizes (\(R_H \approx 18\!-\!20 \, \text{nm}\)) but varying concentrations (\(2.4\!-\!13.9 \, \text{nmol}\,\text{L}^{-1}\)), demonstrating the capability of FCS to quantify diffusion dynamics and sample composition at the nanoscale.
