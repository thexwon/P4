\chapter{Experimental Setup}


%Bin mir nicht sicher was davon hierhin gehört und was zu Theory, können wir ja später noch rumschieben



\section{Light Microscope}

\subsection{Objectives}
Three different objectives are used for the experiments.
\begin{itemize}
    \item \textbf{Nikon E Plan 10x/0.25} uses air as an immersion medium. It has a working distance of \SI{7}{mm} and is used for low resolution imaging.
    \item \textbf{Nikon Plan Apo 60x/1.40 oil} uses oil as an immersion medium. It has a working distance of \SI{280}{\micro m} and is used for high resolution imaging.
    \item \textbf{Leica HCX PL Apo 63x/1.2 w} uses water as an immersion medium. It has a working distance of \SI{300}{\micro m} and is used for FCS measurements.
\end{itemize}


\subsection{Illumination Sources}
Depending on the mode, two different light sources are chosen for the experiments.
\begin{itemize}
    \item A \textbf{He-Ne laser} ($\lambda = \SI{632.8}{nm}$) is used in the confocal mode.
    \item A \textbf{Halogen lamp} is used for conventional wide-field reflection microscopy. It provides a range of different wavelengths, resulting in white light.
\end{itemize}


\subsection{Detection Devices}
Depending on desired sensitivity and dynamic range, one of the following is chosen to detect the incoming light.
\begin{itemize}
    \item A \textbf{photomultiplier tube} (PMT) has a large dynamic range, but comparatively low sensitivity. 
    \item A \textbf{single-photon avalanche diode} (SPAD) has a smaller range, but can detect single photons.
\end{itemize}


\subsection{Galvo-Scanner}
Galvo scanner mirrors are used to raster the laser beam across the sample. They work by using the magnetic field induced by an electric current to produce a mechanical turning motion, which is proportional to the current. 


\subsection{Optical Filters, Beam Splitters}
Optical filters are used to suppress Raman scattering and scattered laser light (Rayleigh scattering). A dichroic beam splitter is used to separate the laser light from the fluorescence light, such that only the fluorescence light is detected.




\section{Confocal Fluorescence Microscope}
Confocal microscopy is based on a 1957 patent by Minsky \cite{patent}, which improves on the resolution of traditional light microscopes by employing point illumination of the sample, and a pinhole in front of the detector to block out light from out of focus sources, as described in section \ref{sec:FCS_conf}, visualized in figure \ref{fig:FCS_patent}. 


\begin{figure}
    \centering
    \includegraphics[width=0.9\linewidth]{Flourescence Correlation (FCS)/include/grafik.png}
    \caption{Drawing of Minskys invention, taken from \cite{patent}. The point-like illumination of the sample and pinhole in front of the detector improve the resolution of the microscope.}
    \label{fig:FCS_patent}
\end{figure}

The confocal volume, i.e. the illuminated sample volume, is usually of the order of a femtoliter, which covers only a small part of the sample. To construct an image of the whole sample, the microscopy needs to scan the sample, usually done in a square grid configuration by moving the sample stage.

Single-molecule detection is possible in a confocal fluorescence microscope because the point-like focused excitation spot and the pinhole drastically reduce the background signal. With few fluorophores present in the femtoliter observation volume, their emission appears as distinct photon bursts that can be detected with sensitive detectors such as Single Photon Avalanche Diodes. To separate fluorescence emission and source light, a dichroic mirror is used. A sketch of a confocal fluorescence microscope setup is shown in figure \ref{fig:FCS_sketch}.

\begin{figure}
    \centering
    \includegraphics[width=0.7\linewidth]{Flourescence Correlation (FCS)//include/SFC_sketch.png}
    \caption{Sketch of a confocal fluorescence microscope setup. The light of the light source excites the fluorescent sample, which emits fluorescence light in all directions. Only a small part of this light is transmitted to the detector. Based on \cite{confocalmic_vid}.}
    \label{fig:FCS_sketch}
\end{figure}
