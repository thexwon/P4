\chapter{Data Analysis and Discussion}

\section{Task 1: Familiarization with Microscope and Control Software}

\subsection{Task 1a}
Using the Nikon E Plan 10x/0.25 objective, which utilizes air as an immersion medium, an EPROM SI memory chip is examined to get familiar with the microscope. Different areas and structures on the chip are looked upon. 

\begin{figure}[h]
    \centering
    \includegraphics[width=0.5\linewidth]{Flourescence Correlation (FCS)/Daten/si_chip_1nm1.png}
    \caption{Image of the EPROM SI memory chip at \SI{1.1}{\mu m} pixel size.}
    \label{fig:si_chip_1.1nm}
\end{figure}



\subsection{Task 1b}
Two different images are taken at $\SI{1.1}{\mu m}$ and $\SI{0.1}{\mu m}$ pixel size. They are depicted in Figs. \ref{fig:si_chip_1.1nm} and \ref{fig:si_chip_0.1nm}. As can be seen, the image at a pixel size of $\SI{0.1}{\mu m}$ looks significantly less distinguished. Looking at the smallest discernible features and the amount of pixels over which the blurring stretches (see Fig. \ref{fig:resolution_power_image_estimation}), the resolving power of the microscope can be estimated to be approximately
$$\text{5 Pixels} \cdot \frac{\SI{0.1}{\mu m}}{\text{Pixel}} = \SI{0.5}{\mu m}$$
This value will later be compared to the one determined via estimation of microscope resolution based on immobilized fluorescent beads.

\begin{figure}[h!]
    \centering
    \includegraphics[width=0.5\linewidth]{Flourescence Correlation (FCS)/Daten/si_chip_0nm1.png}
    \caption{Image of the EPROM SI memory chip at \SI{0.1}{\mu m} pixel size.}
    \label{fig:si_chip_0.1nm}
\end{figure}

\begin{figure}[h!]
    \centering
    \includegraphics[width=0.5\linewidth]{Flourescence Correlation (FCS)/resolution_power_image_estimation.png}
    \caption{Zoomed in section of Fig. \ref{fig:si_chip_0.1nm} to distinguish individual pixels. Based on the number of pixels over which the features of the chip are blurred, the resolution power of the microscope is approximated.}
    \label{fig:resolution_power_image_estimation}
\end{figure}





\section{Task 2: Estimation of Microscope Resolution Based on Immobilized Fluorescent Beads}
For this task, immobilized fluorescent beads are looked upon. Their intensity profiles - in this case across a horizontal line - are analyzed to determine the microscope resolution. This is done by fitting the intensity to a Gaussian model:
\begin{align}
    I(x) = I_0 \cdot e^{-\frac{(x-\mu)^2}{2\sigma^2}}
\end{align}
with the intensity at point $x$ being $I(x)$, the maximum intensity $I_0$, the standard deviation $\sigma$ and the maximum position $\mu$. The standard deviation $\sigma$ is also related to the full width at maximum FWHM via
$$\omega_\text{fwhm} = \sqrt{8 \text{ ln}(2)} \cdot \sigma$$
Using this, the lateral resolution $\omega_\mathrm{lat}$ can be determined via
\begin{equation}
\label{eq:lateral_resolution}
    \omega_\mathrm{lat} = \sqrt{\omega_\mathrm{fwhm}^2 - \omega_\mathrm{bead}^2}
\end{equation}
since the bead size $\omega_\mathrm{bead}$ is known to be $\omega_\mathrm{bead} = \SI{0.04}{\mu m}$. From among the beads in Fig. \ref{fig:beads_numbered}, the ten numbered ones were analyzed and Gaussian fits (Fig. \ref{fig:beads_gaussian_fits}) were applied.

\begin{figure}[h]
    \centering
    \includegraphics[trim=0 2cm 24cm 0, clip, width=0.6\linewidth]{Flourescence Correlation (FCS)/beads_numbered.jpg}
    \caption{Image of multiple beads. The ten numbered beads (red numbers) are chosen for determination of the lateral microscope resolution via the fwhm values of their respective intensity profiles.}
    \label{fig:beads_numbered}
\end{figure}

\begin{figure}[h]
    \centering
    \begin{subfigure}{0.3\textwidth}
        \centering
        \includegraphics[width=\textwidth]{Flourescence Correlation (FCS)/fit_bead_1.jpg}
        %\caption{Bead 1}
    \end{subfigure}
    \hfill
    \begin{subfigure}{0.3\textwidth}
        \centering
        \includegraphics[width=\textwidth]{Flourescence Correlation (FCS)/fit_bead_2.jpg}
        %\caption{Bead 2}
    \end{subfigure}
    \hfill
    \begin{subfigure}{0.3\textwidth}
        \centering
        \includegraphics[width=\textwidth]{Flourescence Correlation (FCS)/fit_bead_3.jpg}
        %\caption{Bead 1}
    \end{subfigure}
    \vspace{0.5cm}
    \begin{subfigure}{0.3\textwidth}
        \centering
        \includegraphics[width=\textwidth]{Flourescence Correlation (FCS)/fit_bead_4.jpg}
        %\caption{Bead 1}
    \end{subfigure}
    \hfill
    \begin{subfigure}{0.3\textwidth}
        \centering
        \includegraphics[width=\textwidth]{Flourescence Correlation (FCS)/fit_bead_5.jpg}
        %\caption{Bead 2}
    \end{subfigure}
    \hfill
    \begin{subfigure}{0.3\textwidth}
        \centering
        \includegraphics[width=\textwidth]{Flourescence Correlation (FCS)/fit_bead_6.jpg}
        %\caption{Bead 1}
    \end{subfigure}
    \vspace{0.5cm}
    \begin{subfigure}{0.3\textwidth}
        \centering
        \includegraphics[width=\textwidth]{Flourescence Correlation (FCS)/fit_bead_7.jpg}
        %\caption{Bead 1}
    \end{subfigure}
    \hfill
    \begin{subfigure}{0.3\textwidth}
        \centering
        \includegraphics[width=\textwidth]{Flourescence Correlation (FCS)/fit_bead_8.jpg}
        %\caption{Bead 2}
    \end{subfigure}
    \hfill
    \begin{subfigure}{0.3\textwidth}
        \centering
        \includegraphics[width=\textwidth]{Flourescence Correlation (FCS)/fit_bead_9.jpg}
        %\caption{Bead 1}
    \end{subfigure}
    \vspace{0.5cm}
    \begin{subfigure}{0.3\textwidth}
        \centering
        \includegraphics[width=\textwidth]{Flourescence Correlation (FCS)/fit_bead_4.jpg}
        %\caption{Bead 1}
    \end{subfigure}
    \caption{Recorded intensities (blue dots) and Gaussian fits (red lines) to the ten beads depicted and numbered in Fig. \ref{fig:beads_numbered}. Each data point corresponds to one pixel and its respective intensity.}
    \label{fig:beads_gaussian_fits}
\end{figure}

\begin{table}[h]
\caption{FWHM values extracted from Gaussian fits for 10 beads.}
\centering
\begin{tabular}{c c}
\toprule
Bead & FWHM [µm] \\
\midrule
1 & 0.36 $\pm$ 0.02 \\
2 & 0.34 $\pm$ 0.02 \\
3 & 0.31 $\pm$ 0.01 \\
4 & 0.39 $\pm$ 0.02 \\
5 & 0.34 $\pm$ 0.02 \\
6 & 0.41 $\pm$ 0.03 \\
7 & 0.36 $\pm$ 0.02 \\
8 & 0.37 $\pm$ 0.01 \\
9 & 0.35 $\pm$ 0.01 \\
10 & 0.41 $\pm$ 0.02 \\
\midrule
Mean & 0.365 $\pm$ 0.006 \\
\bottomrule
\end{tabular}
\label{tab:fwhm_values}
\end{table}

The fwhm values for the ten bead intensities extracted from the fits can be found in table \ref{tab:fwhm_values}, as can the mean of all the values. Mean, median and the uncertainties are also depicted as a box plot in Fig. \ref{fig:boxplot}.

\begin{figure}[h]
    \centering
    \includegraphics[width=0.6\linewidth]{Flourescence Correlation (FCS)/boxplot.jpg}
    \caption{Box plot of the determined fwhm values. Mean and median are closely spaced, since all fwhm values are in a similar range.}
    \label{fig:boxplot}
\end{figure}



Using this and equation \ref{eq:lateral_resolution}, the lateral resolution is determined to be
$$\omega_\mathrm{lat} = \SI{363(6)}{\nano\meter}$$
which is in the same order of magnitude as the rough estimate from Task 1.
It is, however, significantly larger than the expected value. Possible sources of systematic error include grouping of multiple beads (unlikely to result in the observed pattern) or a different size than the one expected.















\section{Task 3: FCS Measurements of Freely Diffusing Fluorescent Molecules and Nanoparticles}
The goal of this task is to determine hydrodynamic radius and concentration of nanoparticles in various samples. In order to extract useful information from each FCS measurement, the effective focal volume needs to be known, which is determined in Section \ref{section:fcs_task3a} using a reference sample of Atto655 with known diffusion constant. With this calibration, the nanoparticle samples are characterized in section \ref{section:fcs_task3b}. 

\subsection{Task 3a: Calibration FCS Measurement Using an Atto655 Sample}
\label{section:fcs_task3a}
In this task, the auto-correlation function of the fluorescence intensity of Atto655 solved in water is recorded and fitted to the function 
\begin{equation}
G(\tau) = G(0) \cdot \left[1 + \frac{\tau}{\tau_D}\right]^{-1} \cdot \left[1 + \frac{\tau}{\tau_D S^2}\right]^{-1/2}
\end{equation}
which is further described in section \ref{section:fcs_brownian}. The fit is done using only data from \SIrange{0.01}{1000}{ms} in order to avoid the signal influence of the triplet state, as it is not accounted for in the fitting function.

\begin{figure}
    \centering
    \includegraphics[width=0.9\linewidth]{Flourescence Correlation (FCS)/fcs_3a.png}
    \caption{Auto-correlation measurement of Atto655 solved in water and the resulting fit.}
    \label{fig:fcs_3a}
\end{figure}

For this task, the parameter $S = 5$ is fixed and the parameters $G(0)$ and $\tau_D$ are determined by fitting the recorded data, shown in figure \ref{fig:fcs_3a}, with the results
\begin{equation*}
    \begin{split}
        G(0) &= \SI{0.0275 \pm 0.0002}{}\\
        \tau_D &= \SI{0.0491 \pm 0.0008}{ms}
    \end{split}
\end{equation*}
with the uncertainties resulting from the fit. From this the lateral width $r_0$ can be determined.
\begin{equation*}
    r_0 = \sqrt{4\cdot D\cdot \tau_D} = \SI{0.272 \pm0.007}{microns}
\end{equation*}
with $D= \SI{378\pm20}{\micro\meter\squared\per\second}$ being the diffusion constant of Atto655 in water at a temperature of $T = \SI{20.5 \pm 0.5}{\degree C}$, same as during the experiment. 
The uncertainty on $D$ results from the read-off uncertainty of the temperature, as well as the $0.19\%$ uncertainty on the tabulated values of $D$ in the preparatory material. The uncertainty on $r_0$ and all following parameters are propagated using gaussian error propagation. \\
\\
The lateral width is of the same order of magnitude as the resolution of \SI{0.365}{microns} determined in task 2. However, this resolution was determined with an oil objective with a numerical aperture of 1.4, while this FCS measurement is done using a water objective with $NA = 1.2$. The appropriate resolution in water can be estimated using $\omega_{\text{oil}}\cdot \frac{NA_{\text{oil}}}{NA_{\text{water}}} =  \SI{0.426}{microns}$. It seems the fluorescent area is smaller than the resolvable area. \\


The effective focal volume can now be determined from
\begin{equation*}
    V_{eff} = \pi^{\frac{3}{2}}r_0^2z_0 = \pi^{\frac{3}{2}}Sr_0^3 = \SI{0.56 \pm 0.05 }{\femto\liter}
\end{equation*}
which is of the expected order of magnitude. This effective focal volume is used as a calibration for the following measurements of nanoparticle samples. 

\subsection{Task 3b: FCS Measurement and Characterization of Nanoparticles}
\label{section:fcs_task3b}
In this task, the hydrodynamic radius and concentration of nanoparticles are determined. The recorded FCS measurements are fitted in the same manner as in section \ref{section:fcs_task3a}, with the fits and measurements of the six samples shown in figure \ref{fig:fcs_3b}, also tabulated in table \ref{table:fcs_3b}.
\begin{figure}
    \centering
    \includegraphics[width=\linewidth]{Flourescence Correlation (FCS)/fcs_3b.png}
    \caption{Auto-correlation measurement different samples of nanoparticles solved in water and the resulting fit.}
    \label{fig:fcs_3b}
\end{figure}

Using the Stokes-Einstein-Theorem from section \ref{sec:fcs_stokes}, the hydrodynamic radius $R_H$ can be determined for each sample using the respective fit results for $G(0)$ and $\tau_D$.
\begin{equation*}
    R_H = \frac{k_B T}{6\pi \eta D}
\end{equation*}
with the Boltzmann constant $k_B$ and $\eta = \SI{989\pm12}{\micro\pascal\second}$ the viscosity of water at a temperature of $T = \SI{20.5 \pm 0.5}{\degree C}$, same as measured during the experiment, given by the preparatory material.

Furthermore, the concentration of nanoparticles per litre in the sample can be calculated using \begin{equation*}
    c = \frac{1}{G(0)V_\text{eff}}
\end{equation*}
For easier display, the concentration per mol $c_\text{mol} = \frac{c}{N_A}$ is calculated and shown in table \ref{table:fcs_3b}, as well as the hydrodynamic radius of each sample.

\begin{table}
    \centering
    
    \caption{Tabulated values of the fit results for $\tau_d$ and $G(0)$, as well as the derived quantities $R_H$ and $c$.}
    \label{table:fcs_3b}
    \begin{tabular}{ccccc}
        \toprule
        Sample &
        $G(0)$ &
        \makecell{Diffusion time \\$\tau_d$ in \si{\milli\second}} &
        \makecell{Hydrodynamic Radius\\  $R_H$  in \si{\nano\meter}} &
        \makecell{Concentration\\ $c_\text{mol}$ in \si{\mole\per\litre}}
        \\
        \midrule
        1 & \SI{0.2968\pm0.0003}{}&\SI{1.548\pm0.009}{} &\SI{1.8\pm 0.1}{} &\SI{10.0\pm0.8}{} \\
        2 & \SI{0.4048\pm0.0009}{}&\SI{1.58\pm0.02}{}   &\SI{1.9\pm 0.1}{} &\SI{7.2 \pm0.6}{}\\
        3 & \SI{1.208\pm0.002}{}&\SI{1.69 \pm0.01}{}    &\SI{2.0\pm 0.1}{} &\SI{2.4 \pm0.2}{} \\
        4 & \SI{0.2119 \pm0.0001}{}&\SI{1.713\pm0.006}{}&\SI{2.0\pm 0.1}{} &\SI{13.9\pm1.1}{} \\
        5 & \SI{0.4614\pm0.0003}{}&\SI{1.690\pm0.007}{} &\SI{2.0\pm 0.1}{} &\SI{6.4 \pm0.5}{} \\
        6 & \SI{0.9042\pm0.0007}{}&\SI{1.632\pm0.008}{} &\SI{1.9\pm 0.1}{} &\SI{3.2 \pm0.3}{} \\
        \bottomrule
    \end{tabular}
\end{table}


It seems the samples consist of nanoparticles of roughly the same size, in varying concentrations. Based on their similar diffusion times, sample one, two and six might contain the same nanoparticles in different concentrations, as well as samples three, four and five.  