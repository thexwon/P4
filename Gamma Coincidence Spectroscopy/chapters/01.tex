\chapter{Theory}

\section{Gamma Decay of Atomic Nuclei}
After radioactive decays such as $\alpha$ or $\beta$ decay, atomic nuclei are often left in excited states. The de-excitation to lower-energy states or to the ground state proceeds via the emission of $\gamma$ radiation. Depending on the nuclear level structure, the transition may occur either directly or through one or more intermediate states. In the latter case, a sequence of successive $\gamma$ emissions, referred to as a $\gamma$ cascade, is observed. Examples of both can be seen in figure \ref{fig:Co_decay}.

\begin{figure}
    \centering
    \includegraphics[width=0.9\linewidth]{Gamma Coincidence Spectroscopy/include/gamma_coincidence_Co_decay.png}
    \caption{Possible decay chains of $^{60}$Co. Figure a) shows direct $\gamma$-transitions into the ground state after $\beta$-decays, while figure b) shows a two part $\gamma$-decay chain into the ground state; a simple gamma cascade. Taken from the provided preparatory material.}
    \label{fig:Co_decay}
\end{figure}

The energies of the emitted $\gamma$ quanta are discrete and correspond to the energy differences between nuclear states. Consequently, the measurement of $\gamma$ energies provides direct information about nuclear levels.

\section{Interaction of Gamma Radiation with Matter}
In the energy range relevant to this experiment, $\gamma$ rays interact with matter mainly through the photoelectric effect and Compton scattering.\\

In the photoelectric effect, the $\gamma$ quantum is completely absorbed by an atomic electron, leading to the formation of a photopeak in the energy spectrum.\\
In Compton scattering, only part of the $\gamma$ energy is transferred to an electron, resulting in a continuous energy distribution up to a sharp cutoff known as the Compton edge, which corresponds to the highest energy that can be transferred to a weakly bound electron of a detector's atom by an incident photon in a single scattering process.\\

Additional spectral features such as backscatter peaks arise from $\gamma$ rays that are scattered in surrounding materials before entering the detector. 

\section{Gamma-Ray Detectors}
Two different detector types are used in this experiment: a NaI(Tl) scintillation detector and a high-purity germanium (HPGe) detector. \\
In a NaI detector, incoming $\gamma$ rays produce scintillation light, which is converted into an electrical signal by a photomultiplier tube. NaI detectors are characterized by high detection efficiency but limited energy resolution.\\
HPGe detectors operate as semiconductor detectors, where $\gamma$ rays create electron–hole pairs in the germanium crystal. Due to the small energy required to generate a charge carrier pair, HPGe detectors achieve superior energy resolution. However, they require cooling to liquid nitrogen temperatures to reduce thermal noise.

\section{Energy Resolution and Calibration}
The energy resolution of a detector describes its ability to distinguish between $\gamma$ rays of different energies and is commonly quantified by the full width at half maximum (FWHM) of a photopeak. The energy resolution depends on statistical fluctuations in the signal generation process and generally improves with increasing $\gamma$ energy.

Energy calibration is performed by assigning known $\gamma$ energies from calibration sources to the corresponding photopeak positions in the measured spectra. A linear relation between channel number and energy is assumed over the relevant energy range.

\section{Coincidence Spectroscopy}
Coincidence spectroscopy is a method for studying correlated $\gamma$ emissions. Both gamma-ray detectors are operated simultaneously, and events are considered coincident if signals occur in both detectors within a predefined time window. True coincidences originate from correlated transitions within a $\gamma$ cascade, while random coincidences result from unrelated events occurring within the coincidence window.
By applying appropriate time gates to the recorded time spectrum, true coincidences can be enhanced relative to random coincidences. This technique allows the identification of $\gamma$ cascades and the reconstruction of nuclear level schemes.
