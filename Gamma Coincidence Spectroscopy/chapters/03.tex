\chapter{Experimental Procedure}

\section{Task 1: Energy Calibration}
In the first step, the energy calibration is performed. For this purpose, the $\gamma$ spectra of the well-known calibration sources $^{22}$Na, $^{137}$Cs, and $^{57}$Co are recorded with both detectors. The positions of the photopeaks are determined and assigned to the known $\gamma$ energies. From the peak widths, the energy-dependent energy resolution of the detectors is obtained.

\section{Task 2: Time Calibration}
Subsequently, the time calibration is carried out. Using the $^{22}$Na source, the time spectrum is recorded while defined delays are set in the delay unit. From the shift of the coincidence peak, the time axis of the TAC is calibrated. Afterwards, the delay is adjusted such that the coincidence peak is located approximately in the center of the ADC spectrum.

\section{Task 3: $^{60}$Co $\gamma$-Spectrum}
In the next step, the $\gamma$ spectrum of the $^{60}$Co source is recorded with both detectors. The energies of the photopeaks as well as the positions of the Compton edges and backscatter peaks are determined and interpreted.

\section{Tasks 4 \& 5: Analysis}
This is followed by the coincidence analysis of the $^{60}$Co data. Using energy gates and different time gates, true and random coincidences are analyzed separately. In particular, it is investigated which $\gamma$-transitions occur in coincidence. Based on these results, the underlying level scheme of the decay is derived, and it is examined whether additional coincidences can be observed.

Finally, the dependence of the energy resolution on the $\gamma$ energy is investigated for both detector types, and the difference between the NaI and Ge detectors is qualitatively explained.
