\chapter{Analysis}
All the absorption peaks are fitted using a Gaussian model in combination with background. For FePO$_4$ and FeSO$_4$, a model with two combined Gauss peaks is used. For natural iron, each peak is fitted individually due to problems that would occur when fitting with too many parameters at once. 

\section{Velocity Calibration}

To determine the velocity of the source, a Michelson interferometer is used, where the usually moveable mirror is attached to the moving source, as seen in figure \ref{fig:moessbauer_setup}. The velocity can then be measured by counting the amount of intensity maxima occurring at the photodiode of the interferometer. The frequency of these maxima is given by the velocity of the source $v$ and the laser wavelength $\lambda = \SI{650}{nm}$ of the interferometer, which can be directly expressed by the amount of recorded interference maxima $N$ during the time $T = \frac{10}{1024} \si{s}$, the total amount of time in which one of 1024 channels is active. 
\begin{equation*}
    \begin{split}
        f &= \frac{2v}{\lambda} = \frac{N}{T} \\
        \rightarrow v &= \frac{\lambda}{2}\frac{N}{T} = N \cdot 0.0328 \SI{}{\frac{mm}{s}}
    \end{split}
\end{equation*}

\begin{figure}
    \centering
    \includegraphics[width=0.6\linewidth]{Mößbauer Effect/VelocityCalibrationSpectrum.png}
    \caption{Velocity calibration measurement, with maxima count $N$ measured within a total measurement time of \SI{10}{s} in each channel $c$. }
    \label{fig:mossbauer_vecolitycalibmeas}
\end{figure}
A calibration run with a total measurement time of \SI{10}{s} was performed, resulting in the calculated velocity for each channel as shown in figure \ref{fig:mossbauer_vecolitycalibmeas}. As the amount of passed maxima at the interferometer is the same for positive and negative velocities of equal speed, the velocities corresponding to the source moving away from the detector needed to be inverted, as shown in figure \ref{fig:mossbauer_vecolitycalibfit}

\begin{figure}
    \centering
    \includegraphics[width=0.6\linewidth]{Mößbauer Effect/VelocityCalibrationFit.png}
    \caption{Velocity calibration fit, relating the channel $c$ to a velocity $v$.}
    \label{fig:mossbauer_vecolitycalibfit}
\end{figure}
Based on the equation for the velocity, a linear fit was applied to both sides of the spectrum, resulting in the linear equations

\begin{numcases}{v=}
    $\SI{-4.856(3)e-5}{\frac{m}{s}c + \SI{0.01245(1)}{\frac{m}{s}}} $   &  for $0 \leq c < 512$\\
    $\SI{4.857(3)e-5}{\frac{m}{s}c + \SI{-0.03733(2)}{\frac{m}{s}}} $   & for  $512 \leq c <1024$
\end{numcases}
which can now be used to relate channels to velocities. The uncertainty on the number of counts of maxima was estimated as 1, resulting in the uncertainties of $v$. The maximum speed is roughly \SI{12.5}{\frac{mm}{s}}.


\section{Vacromium}

\begin{figure}[htbp]
    \centering
    \begin{minipage}[b]{0.48\textwidth}
        \centering
        \includegraphics[width=\textwidth]{Mößbauer Effect/vacromium_fit1.png}
    \end{minipage}
    \hfill
    \begin{minipage}[b]{0.48\textwidth}
        \centering
        \includegraphics[width=\textwidth]{Mößbauer Effect/vacromium_fit2.png}
    \end{minipage}
    \caption{Lorentzian fits for vacromium. The guess which the fit is based on is shown in green, the guessed peak position is marked with a vertical red line. The fit (red) also includes a background assumed to be linear.}
    \label{fig:vacromium_fit}
\end{figure}

Vacromium (stainless steel) exhibits a single resonance absorption peak. The seperate fits for both velocity sweeps are depicted in Fig. \ref{fig:vacromium_fit}. A Gauss peak was fitted to the data. Since the peak was already distinguishable by eye, its position was chosen for a guess of the fit function. Converting the peak positions into velocity, combining them and calculating the respective energy yields

\begin{equation}
    E_0 = \SI{-11.9(0.3)}{neV}
\end{equation}

The width 

\begin{equation}
    \Gamma = \SI{590.1(8)}{neV}
\end{equation}

of a Lorentzian peak can be used to calculate the lifetime of the excited state:

\begin{equation}
    \tau = \frac{\hbar}{\Gamma} = \SI{1.115(2)}{ns}
\end{equation}

This value is two orders of magnitude smaller than the expected one. Calculation via the width of a Gaussian distribution instead of a Lorentzian gave similar values. One possible explanation might be the comparatively high noise in the data. As can be seen from many of the fits, even peaks that should have the same width according to theory, exhibit high deviations from one another (roughly up to 100\%). This still doesn't explain two orders of magnitude. Another explanation could be that the measurement itself has a strong uncertainty associated to it, which could cause the absorption peaks to widen significantly.







\section{FePO$_4$}

\begin{figure}[htbp]
    \centering
    \begin{minipage}[b]{0.48\textwidth}
        \centering
        \includegraphics[width=\textwidth]{Mößbauer Effect/fepo4_fit1.png}

    \end{minipage}
    \hfill
    \begin{minipage}[b]{0.48\textwidth}
        \centering
        \includegraphics[width=\textwidth]{Mößbauer Effect/fepo4_fit2.png}
    \end{minipage}
    
    \caption{Lorentzian fits for FePO$_4$. The guess which the fit is based on is shown in green. The fit (red) also includes a background assumed to be linear.}
    \label{fig:fepo4_fit1}
\end{figure}

What distinguishes these absorption peaks is their overlapping. Nonetheless, the fits as depicted in figure \ref{fig:fepo4_fit1} were able to decently capture the behaviour. The absorption positions in terms of energy are determined to be
\begin{align}
    E_0 = \SI{-8.1(1.4)}{neV}
    E_1 = \SI{28.8(2.2)}{neV}
\end{align}

The isomeric shift is thus given by

\begin{equation}
    I = \frac{E_0 + E_1}{2} = \SI{10.3(1.2)}{neV}
\end{equation}

Since the quadrupole moment $Q$ is known to be $Q = \SI{0.21(1)e-28}{m^2}$, the second derivative of the electric potential at the nucleus can be calculated.

\begin{equation}
    \partial_z^2V = \frac{2(E_1 - E_0)}{Q} = \SI{3.51(23)e21}{Vm^{-2}}
\end{equation}







\section{FeSO$_4$}

\begin{figure}[htbp]
    \centering
    \begin{minipage}[b]{0.48\textwidth}
        \centering
        \includegraphics[width=\textwidth]{Mößbauer Effect/feso4_fit1.png}
    \end{minipage}
    \hfill
    \begin{minipage}[b]{0.48\textwidth}
        \centering
        \includegraphics[width=\textwidth]{Mößbauer Effect/feso4_fit2.png}
    \end{minipage}
    \caption{Lorentzian fits for FeSO$_4$. The guess which the fit is based on is shown in green. The fit (red) also includes a background assumed to be linear.}
    \label{fig:feso4_fit1}
\end{figure}

In a similar fashion to FePO$_4$, two absorption peaks were observed for FeSO$_4$. However, in this case the peaks are much more separated, making the fitting depicted in figure \ref{fig:feso4_fit1} more reliable. The absorption positions in terms of energy are determined to be

\begin{align}
    E_0 = \SI{-13.0(1.2)}{neV}
    E_1 = \SI{120.5(4)}{neV}
\end{align}

The isomeric shift is thus given by

\begin{equation}
    I = \frac{E_0 + E_1}{2} = \SI{53.7(6)}{neV}
\end{equation}

Since the quadrupole moment $Q$ is known to be $Q = \SI{0.21(1)e-28}{m^2}$, the second derivative of the electric potential at the nucleus can be calculated.

\begin{equation}
    \partial_z^2V = \frac{2(E_1 - E_0)}{Q} = \SI{1.27(1)e22}{Vm^{-2}}
\end{equation}





\section{Natural Iron}

\begin{figure}[h!]
    \centering
    \begin{minipage}[b]{0.48\textwidth}
        \centering
        \includegraphics[width=\textwidth]{Mößbauer Effect/iron_fit1.png}
        
    \end{minipage}
    \hfill
    \begin{minipage}[b]{0.48\textwidth}
        \centering
        \includegraphics[width=\textwidth]{Mößbauer Effect/iron_fit2.png}
        
    \end{minipage}
    \caption{Gaussian fits for iron. The guess which the fit is based on is shown in green, the guessed peak positions are marked with vertical red lines. The fit (red) also includes a background assumed to be linear.}
    \label{fig:iron_fit2}
\end{figure}

Combining all the fits depicted in figure \ref{fig:iron_fit2} results in the following peak positions:
\begin{itemize}
    \item Peak 1: \SI{255(1)}{neV}
    \item Peak 2: \SI{143(1)}{neV}
    \item Peak 3: \SI{35(1)}{neV}
    \item Peak 4: \SI{-51(1)}{neV}
    \item Peak 5: \SI{-157(1)}{neV}
    \item Peak 6: \SI{-267(1)}{neV}
\end{itemize}
By analyzing the peak positions in relation to one another and the zero-energy position, the isoshift is determined to be \SI{-6.78(35)}{neV}. The magnetic field can be calculated to be
\begin{equation}
    B = \SI{-33.86(31)}{T}
\end{equation}
Assuming $\mu_\mathrm{g} = 90.3(7)\cdot 10^{-3} \mu_\mathrm{N}$, $I_\mathrm{g} = \frac{1}{2}$ and $I_\mathrm{a} = \frac{3}{2}$, the magnetic moment of the exited state $\mu_\mathrm{a}$ can be calculated:
\begin{equation}
    \mu_\mathrm{a} = \SI{-4.92(05)}{neV/T} = -0.156(2) \mu_\mathrm{k} = -1.73(1) \mu_\mathrm{g}
\end{equation}


    


