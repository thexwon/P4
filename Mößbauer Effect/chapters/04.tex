\chapter{Analysis}
All the absorption peaks are fitted using a Gaussian model in combination with background. For FePO$_4$ and FeSO$_4$, a model with two combined Gauss peaks is used. For natural iron, each peak is fitted individually due to problems that would occur when fitting with to many parameters at once. 

\begin{figure}[htbp]
    \centering
    \begin{minipage}[b]{0.48\textwidth}
        \centering
        \includegraphics[width=\textwidth]{Mößbauer Effect/vacromium_fit1.png}
        \label{fig:vacromium_fit1}
    \end{minipage}
    \hfill
    \begin{minipage}[b]{0.48\textwidth}
        \centering
        \includegraphics[width=\textwidth]{Mößbauer Effect/vacromium_fit2.png}
        \label{fig:vacromium_fit2}
    \end{minipage}
    \caption{Gaussian fits for vacromium. The guess which the fit is based on is shown in green. The fit (red) also includes a background assumed to be linear.}
\end{figure}






\section{FePO$_4$}
\begin{figure}[htbp]
    \centering
    \begin{minipage}[b]{0.48\textwidth}
        \centering
        \includegraphics[width=\textwidth]{Mößbauer Effect/fepo4_fit1.png}
        \label{fig:fepo4_fit1}
    \end{minipage}
    \hfill
    \begin{minipage}[b]{0.48\textwidth}
        \centering
        \includegraphics[width=\textwidth]{Mößbauer Effect/fepo4_fit2.png}
        \label{fig:fepo4_fit2}
    \end{minipage}
    \caption{Gaussian fits for FePO$_4$. The guess which the fit is based on is shown in green. The fit (red) also includes a background assumed to be linear.}
\end{figure}





\section{FeSO$_4$}
\begin{figure}[htbp]
    \centering
    \begin{minipage}[b]{0.48\textwidth}
        \centering
        \includegraphics[width=\textwidth]{Mößbauer Effect/feso4_fit1.png}
        \label{fig:feso4_fit1}
    \end{minipage}
    \hfill
    \begin{minipage}[b]{0.48\textwidth}
        \centering
        \includegraphics[width=\textwidth]{Mößbauer Effect/feso4_fit2.png}
        \label{fig:feso4_fit2}
    \end{minipage}
    \caption{Gaussian fits for FeSO$_4$. The guess which the fit is based on is shown in green. The fit (red) also includes a background assumed to be linear.}
\end{figure}


\section{Natural Iron}
\begin{figure}[htbp]
    \centering
    \begin{minipage}[b]{0.48\textwidth}
        \centering
        \includegraphics[width=\textwidth]{Mößbauer Effect/iron_fit1.png}
        \label{fig:iron_fit1}
    \end{minipage}
    \hfill
    \begin{minipage}[b]{0.48\textwidth}
        \centering
        \includegraphics[width=\textwidth]{Mößbauer Effect/iron_fit2.png}
        \label{fig:iron_fit2}
    \end{minipage}
    \caption{Gaussian fits for iron. The guess which the fit is based on is shown in green. The fit (red) also includes a background assumed to be linear.}
\end{figure}
