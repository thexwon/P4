\chapter{Experimental Setup}

A schematic of the measurement setup is depicted in Fig. \ref{fig:moessbauer_setup}. The variation of source speed which is required for the Mössbauer effect is achieved by using drive coils. The speed is tuned and recorded with a Michaelson interferometer and measurement coils. Thus, the frequency of the \SI{14.4}{keV} gammas is slightly shifted by the Doppler effect. Since the source speed is recorded, this shift can be precisely calculated. A Geiger-counter is used to measure the intensity of gammas passing through the sample, allowing for conclusions about the absorption rate and thus the absorption spectrum of the target material.

\begin{figure}
    \centering
    \includegraphics[width=0.5\linewidth]{Mößbauer Effect/moessbauer_setup.png}
    \caption{Schematic of the Mössbauer drive system / Mössbauer velocity transducer (MVT). The laser interferometer is used for speed calibration. The combination of absorber and counter tube is used for detection of the \SI{14.4}{keV} photons emitted by the $^{57}$Co source.}
    \label{fig:moessbauer_setup}
\end{figure}