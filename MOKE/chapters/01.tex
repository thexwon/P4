If not mentioned otherwise, this preparation is based on the MOKE introduction of Physikalisches Institut, Karlsruhe Institute of Technology.
\chapter{Aim of the Experiment and Theoretical Basis}
The primary aim of these experiments is to investigate the magnetic properties of various ferromagnetic thin films and multilayers using the Magneto-optic Kerr Effect (MOKE). For this, we will: 
\begin{itemize}
    \item Measure magnetic hysteresis loops to determine magnetic parameters such as the intrinsic coercivity $H_C$, remanence $M_r$ and saturation magnetization $M_S$. (All tasks)
    \item Characterize the magnetic anisotropy (i.e. easy and hard axes) of samples by employing both polar and longitudinal MOKE geometries. (Task 1 \& 2)
    \item Connect the observed magnetic behaviour with the samples' microstructure, such as crystal orientation, layer composition and sample orientation. (Task 2 \& 4)
\end{itemize}

\section{Ferromagnetism and Hysteresis Behaviour}
\section{Magnetic Anisotropy}
\section{Magneto-optic Kerr Effect (MOKE)}