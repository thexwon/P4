\addcontentsline{toc}{chapter}{Versuchsvorbereitung}

%hab jtz erstmal was von Perplexity genommen als Grundlage und würd das abändern
\chapter*{Versuchsvorbereitung}
\section{Physical Background}

\subsection{Magneto-optic Kerr Effect}

The magneto-optic Kerr effect (MOKE) describes the change in intensity or polarization direction of an incident laser beam after reflection from a ferromagnetic metallic surface. The effect arises from the interaction of polarized light with magnetized materials and is particularly sensitive to thin films - even a few monolayers produce measurable signals.

The Kerr rotation angle \(\theta_K\) and ellipticity \(\epsilon_K\) are defined through the complex Kerr angle \(\Phi_K = \theta_K + i\epsilon_K\). For perpendicular incidence and magnetization, the complex Kerr angle relates to the Voigt constant \(Q\) and refractive index \(n_0\) as:
\[
\Phi_K = i n_0 Q \frac{1}{n_0^2 - 1}
\]

\subsection{Microscopic Origin}

The microscopic origin lies in the spin-orbit interaction combined with exchange splitting in ferromagnets. The dielectric tensor for magnetized media contains off-diagonal elements proportional to magnetization. For circularly polarized light, the refractive indices differ:
\[
n_\pm \approx n_0 (1 \pm \frac{1}{2} \mathbf{u} \cdot \mathbf{Q})
\]
where \(\mathbf{u}\) is the propagation direction unit vector.

Optical transitions between spin-split d-bands and p-bands obey selection rules \(\Delta l = \pm 1\) and \(\Delta m_l = \pm 1\). The exchange energy \(\Delta_{ex}\) (1-2 eV) and spin-orbit energy \(\Delta_{so}\) (tens of meV) create different absorption for left and right circularly polarized light, resulting in magnetic circular dichroism.

\subsection{MOKE Geometries}

Three geometries exist depending on magnetization direction:

\textbf{Polar MOKE:} Magnetization perpendicular to surface (\(M \perp\) surface). Measures out-of-plane magnetization component.

\textbf{Longitudinal MOKE:} Magnetization parallel to surface and in optical plane (\(M \parallel\) surface, in plane of incidence). Measures in-plane magnetization component parallel to beam plane.

\textbf{Transverse MOKE:} Magnetization parallel to surface but perpendicular to optical plane. Measures reflectivity changes (not used in this experiment).

\section{Magnetic Properties}

\subsection{Ferromagnetism}

Ferromagnetic materials exhibit spontaneous magnetization due to exchange interaction with energy:
\[
E_{ij}^{ex} = -2 J_{ij} \mathbf{S}_i \cdot \mathbf{S}_j
\]
For \(J_{ij} > 0\), neighboring spins align parallel. Materials form magnetic domains to minimize magnetostatic energy.

\subsection{Hysteresis Parameters}

\textbf{Saturation magnetization} \(M_S\): Maximum magnetization achieved in fields exceeding saturation field \(H_s\).

\textbf{Remanence} \(M_r\): Magnetization remaining at zero field after saturation.

\textbf{Coercivity} \(H_c\): Reverse field required to reduce magnetization to zero.

Materials are classified as soft (\(\mu_0 H_c < 1\) mT) or hard (\(\mu_0 H_c > 0.1\) T) magnets.

\subsection{Magnetic Anisotropy}

Magnetocrystalline anisotropy arises from spin-orbit interaction coupling spin to crystal lattice. Energy depends on magnetization direction relative to crystal axes:

\textbf{Easy axis:} Direction of spontaneous magnetization alignment (lowest energy).

\textbf{Hard axis:} Direction requiring external field to align magnetization (highest energy).

For thin films, shape anisotropy dominates: demagnetization factor \(N_d = 0\) for in-plane magnetization, \(N_d = 1\) for out-of-plane. Magnetostatic energy:
\[
E_{ms} = \frac{\mu_0}{2} N_d M_S^2
\]
This typically favors in-plane magnetization unless interface effects (e.g., Co/Pt multilayers) overcome it.

\section{Sample Properties}

\subsection{Iron Films}

Iron has bcc structure (lattice constant 0.2866 nm) with easy axis \(\langle 100 \rangle\), medium axis \(\langle 110 \rangle\), hard axis \(\langle 111 \rangle\).

\textbf{Sample 1:} Polycrystalline Fe on amorphous SiO\(_2\) - weak in-plane anisotropy.

\textbf{Sample 2:} Fe(211) on MgO(110) - epitaxial growth with specific in-plane anisotropy.

\textbf{Sample 4:} Fe(100) rotated 45° on MgO(100) - epitaxial with fourfold in-plane symmetry.

\subsection{Cobalt Films}

Cobalt has hcp structure (a = b = 0.2505 nm, c = 0.4089 nm) with easy axis along c-direction (out-of-plane in hcp), hard axis in basal plane.

\textbf{Sample 6:} Co(110) on MgO(100) - easy and hard axes in-plane.

\textbf{Samples 7 and 8:} Co/Pt multilayers - interface-induced perpendicular magnetic anisotropy overcomes shape anisotropy, creating out-of-plane easy axis. Enhanced on MgO(111) substrate.

\section{Experimental Procedure}

\subsection{Setup Components}

Laser diode (638 nm, red), polarizer P and analyzer, polarizing beam splitter (PBS), two photodiodes measuring intensities \(I_1\) and \(I_2\), electromagnet with bipolar current source, Hall probe for field measurement, computer control system "MOKE".

\subsection{Longitudinal MOKE Setup}

Light passes polarizer X (intensity control), iris, adjustable polarizer P (75 units = 1000 mdeg), reflects from sample, two mirrors, enters detector. PBS splits into orthogonal components measured by photodiodes. Set P to \(\pm 45°\) to achieve \(I_1 \approx I_2\) (i.e., \(\Delta I = I_1 - I_2 \approx 0\)).

\subsection{Polar MOKE Setup}

Light passes polarizer P (50 units = 1000 mdeg), non-polarizing beam splitter, through magnet pole piece hole, reflects from sample, deflected by beam splitter into detector. Similar balancing procedure as l-MOKE.

\subsection{Calibration}

Rotate polarizer P by known angle \(\alpha\) (mdeg), measure signal change. Calibration factor:
\[
\text{Calib} = \frac{\alpha}{\Delta \text{Signal}}
\]
Kerr asymmetry relates to rotation angle:
\[
A_K = \frac{I_+ - I_-}{I_+ + I_- - 2I_0'} \approx 2\theta_K / \beta
\]
for small Kerr angle and analyzer angle \(\beta\)