\chapter{Experimental Setup and Measurement principle}
\section{Experimental Geometries}

The MOKE can be measured in various geometries, which are defined by the relative orientation of the magnetization $M$, the sample surface and the plane of incidence, which contains the incident and reflected beams. Possible variations are shown in figure \ref{fig:geoms} and described further below.

\begin{itemize}
    \item Longitudinal MOKE (l-MOKE): In this case, the magnetization $M$ lies within the sample plane and is parallel to the plane of incidence. In this longitudinal geometry, a Kerr rotation proportional to the in-plane component of magnetization along this specific direction can be measured.
    \item Polar MOKE (p-MOKE): The magnetization $M$ is perpendicular to the sample plane. This geometry is sensitive to the out-of-plane component of magnetization. A Kerr rotation proportional to the z-component of the Voigt vector can be measured.
    \item Transversal MOKE (t-MOKE): The magnetization $M$ lies within the sample plane, but is perpendicular to the plane of incidence. This geometry cannot be used to measure a Kerr rotation. Instead, a change in reflectivity depending on the magnetization can be measured. This geometry is not used in the following experimental setup.
    
    
\end{itemize}





\begin{figure}
    \centering
    \includegraphics[width=0.8\linewidth]{MOKE/include/moke geoms.png}
    \caption[Possible geometries for measuring MOKE]{Possible geometries for measuring MOKE. Taken from the provided preparatory material.}
    \label{fig:geoms}
\end{figure}


\section{Experimental Setup}
\begin{figure}
    \centering
    \includegraphics[width=0.9\linewidth]{MOKE/include/moke setup.png}
    \caption[Experimental setup of the longitudinal and polar MOKE]{Experimental setup of the longitudinal and polar MOKE. Taken from the provided preparatory material.}
    \label{fig:setup}
\end{figure}
The apparatus, schematically depicted in Figure \ref{fig:setup}, consists of the following key components: 
\begin{itemize}
    \item Light sources: Laser diodes to act as light sources, emitting red light at $\lambda = \SI{638}{nm}$ with $\SI{1}{mW}$ of power. 
    \item Polarization optics: A polarizer (X) for intensity attenuation in the l-MOKE setup and rotatable polarizers (P) to set the initial polarization states of both l-MOKE and p-MOKE geometries.
    \item Magnet: A bipolar DC electromagnet, driven by a programmable current source. The magnetic field is measured using a Hall probe. It also includes a stage for the samples.
    \item Detector: A MOKE detector box housing the Polarizing Beam Splitter (PBS), two silicon photodiodes, and electronics for amplifying the signals $I_1$ and $I_2$.
    \item Beam guides: Contrary to the figure, only one detector box is used during the experiments. For the l-MOKE, the beam can be steered by two adjustable mirrors into the detector box. P-Moke instead has a lens on an x-y translation stage for alignment and a non-polarizing beam splitter to direct the reflected beam.
    
\end{itemize}

\section{Measurement Principle}

The experimental setup is designed to detect the small Kerr rotation, which is proportional to the sample's magnetization. The core principle is a polarization bridge technique: \\
Light from a laser diode is linearly polarized by an adjustable polarizer (P). This polarized beam is reflected from the magnetized sample, where its polarization undergoes a slight Kerr rotation,$\theta_K$. \\The reflected beam is then directed into a Polarizing Beam Splitter (PBS), which resolves it into two orthogonal, linearly polarized components. These are measured by two separate photodiodes, yielding intensities $I_1$ and $I_2$.\\ Initially, the polarizer (P) is adjusted so that the beam incident on the PBS is oriented at $45\degree$ to its axes, resulting in $I_1 = I_2$ for a demagnetized sample. A Kerr rotation due to magnetization unbalances this condition. The normalized difference signal,
$$\mathrm{Signal} = \frac{I_1-I_2}{I_1+I_2} \propto \theta_K \propto M$$
is therefore directly proportional to the component of the sample's magnetization along the measured axis. By sweeping an external magnetic field generated by the electromagnet and recording this normalized signal, the magnetic hysteresis loop ($M$ against $H$) can be plotted.