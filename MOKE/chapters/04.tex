\chapter{Evaluation}

The before described calibration is carried out for both the p-MOKE and l-MOKE setup, yielding the  calibration factors $(\frac{\Delta\phi}{\Delta S})_{\text{p-MOKE}} =  \SI{1709.4}{\frac{m\degree}{V}}$ and $(\frac{\Delta\phi}{\Delta S})_{\text{l-MOKE}} =\SI{2898.5}{\frac{m\degree}{V}}$. 

\section{Task 1 \& 2: Co thin film and Co/Pt Multilayers (p-MOKE and l-MOKE)}

For each of the samples \#6 (Co thin film), \#7 and \#8 (Co/Pt multilayers), hysteresis curves were recorded in both p-MOKE and l-MOKE configurations, as shown in Fig.~\ref{fig:task12}. 

\subsection*{Determination of Easy Axis of Magnetization}

The easy axis of magnetization can be determined by comparing the hysteresis loops obtained in the two geometries. 
In the geometry where the magnetization aligns more readily with the applied magnetic field, the Kerr signal saturates at smaller magnetic fields and exhibits a square, nearly rectangular hysteresis loop. 
This direction corresponds to the easy axis of magnetization. 
In contrast, along the hard axis the magnetization rotates gradually toward the field direction, resulting in a slanted loop with a much smaller Kerr amplitude and weaker hysteresis.

For thin ferromagnetic films such as pure Co layers, the shape anisotropy usually dominates, leading to an in-plane easy axis.  
However, in Co/Pt multilayers, strong interfacial spin–orbit coupling at the Co/Pt interfaces introduces an additional perpendicular magnetic anisotropy (PMA) that can overcome the shape anisotropy, reorienting the easy axis out of the plane~\cite{multilayers}.  \\

This behaviour is reflected in the recorded hysteresis: For the thin cobalt film sample \#6, the easy magnetic axis lies in the plane. A hard axis cannot be determined from the recorded hysteresis because of the small magnetic field range, but should also lie in the film plane. 
For the Cobalt-Platinum multilayers, the easy axis points out of the plane, as seen in their p-MOKE hysteresis, while the hard axis stays inside the film plane, visible in their l-MOKE hysteresis. \\

\begin{figure}
    \centering
    \includegraphics[width=\linewidth]{MOKE/task12.png}
    \caption{P-MOKE and l-MOKE measurements of samples \#6, \#7 and \#8.}
    \label{fig:task12}
\end{figure}

From the hysteresis loops, the coercive fields $H_c$  and saturation Kerr rotations $\theta_K^{\mathrm{sat}}$ are determined for each geometry (see Table~\ref{tab:task12}), by averaging of the read values whenever possible.  
These quantities allow a direct comparison of the magnetic anisotropy and the influence of the substrate.


\begin{table}[]
    \centering
    \begin{tabular}{lccccc}\toprule
         Sample & Geometry & $\mu_0 H_c$ in mT & $\theta_K^{\mathrm{rem}}$ in m\degree & $\theta_K^{\mathrm{sat}}$ in m\degree & Easy Axis \\\midrule
        \#6 (Co on MgO (100)) & l-MOKE& 8.5&6.5 & 7.5& in-plane\\
             & p-MOKE& -&- &- & -\\
        \#7 (Co/Pt on SiO$_2$) & l-MOKE& 92 & 74& 52.5& out-of-plane\\
             & p-MOKE& 9.5& 82.2& 83.2& \\
        \#8 (Co/Pt on MgO (111)) & l-MOKE& 375 & 82.5& 65& out-of-plane\\
             & p-MOKE& 48 & 75&73 & -\\

         
         
         \bottomrule
    \end{tabular}
    \caption{Summary of determined magnetic parameters for Co thin film and Co/Pt multilayers.}
    \label{tab:task12}
\end{table}

\subsection*{Discussion}

\paragraph{Sample \#6: Cobalt thin film on MgO (100)}
Cobalt crystallizes in a hexagonal closed packed structure with in-plane easy axes due to its magnetic anisotropy. For the Co film on MgO(100), both easy and hard axes lie in the film plane~\cite{prep}.  
The recorded l-MOKE hysteresis loop exhibits a soft magnetic behaviour with low coercivity (\(\mu_0 H_c \approx 10\) mT) and magnetization reversal, consistent with in-plane magnetization.  \\
The p-MOKE signal is similar for magnetization and demagnetization, without any hysteresis. The out-of-plane magnetization component therefore seems to be negligible, indicating both the easy and hard axis lie in the film plane. 

\paragraph{Sample \#7: Cobalt-Platinum multilayer on SiO$_2$}
The multilayer structure introduces interface anisotropy through the Co/Pt interfaces, leading to perpendicular magnetic anisotropy \cite{multilayers}. \\
The p-MOKE hysteresis of sample \#7 shows a square loop with sharp switching, which is characteristic for an easy axis.
The l-MOKE hysteresis shows a weaker, slanted response to the magnetic field, indicating a hard axis in-plane.

\paragraph{Sample \#8: Cobalt-Platinum multilayer on Mg(111)}
The crystallographic alignment of the multilayers and the substrate should enhance the anistropy between in-plane and out-of-plane magnetization. 
As a result, the out-of-plane easy axis becomes even more pronounced, with larger coercivity and squarer hysteresis compared to the amorphous SiO$_2$ substrate.  
This indicates that the crystalline order of the substrate further reinforces the perpendicular anisotropy.\\\\


The comparison between p-MOKE and l-MOKE shows the transition from in-plane to out-of-plane magnetization when moving from a Co thin film to Co/Pt multilayers.  
While shape anisotropy dominates in the single Co film, interface-induced anisotropy in the Co/Pt multilayers seems to dominate the magnetic behaviour.  
The choice of substrate also seems to affects the magnitude of this anisotropy, with crystalline MgO(111) promoting stronger perpendicular magnetic anisotropy than amorphous SiO$_2$. 



\section{Task 3 \& 4: Fe Thin Films (l-MOKE, Anisotropy)}

\begin{figure}
    \centering
    \includegraphics[width=0.6\linewidth]{MOKE/include/task4_energy_fit_offset.png}
    \caption{Polar plot of coersive field over sample orientation angle (black dots). The energy-based fit (red line) shows a weak uniaxial anisotropy associated with the [100] direction superimposed on the otherwise 4-fold symmetric cubic anisotropy.}
    \label{fig:polar_plot}
\end{figure}

To investigate the magnetocristalline anisotropy (see sec. \ref{sec:anisotropy}) of Fe(100), sample \#4 is used. It consists of Fe(100) rotated by 45° in MgO(100). L-MOKE measurements are carried out at different angles in 30° steps to determine the angular dependence of the Kerr-angle. Hysteresis loops at each angle are evaluated by eye to determine the respective coercive field. A plot of coercive field $H_\mathrm{c}$ over the orientation angle can be seen in Fig. \ref{fig:polar_plot}. 
\subsection*{Fit Model}
Additionally, a fit model is used to describe the somewhat symmetric behaviour. Due to the bcc structure of the Fe(100) lattice, a cubic anisotropy field is expected. A superimposed uniaxial anisotropy can also be expected for the Fe/MgO film \cite{PhysRevB.80.094416}. To model the combined field, one can assume the coersive field $H_\mathrm{c}$ to be proportional to the magnetic free energy of the film, which in turn is given by \cite{10.1063/1.2834448, PhysRevLett.90.107201, Won_2013, Bac2015}

\begin{equation}
H_\mathrm{c} = \frac{E}{M} = \frac{1}{8}H_\mathrm{cubic} \cos^2(2(\varphi - \varphi_0)) + \frac{1}{2}H_\mathrm{uniaxial}^{[100]} \sin^2\left(\varphi - \varphi_0 - \frac{3\pi}{4}\right) - H \cos(\varphi - \varphi_H) + H_\mathrm{offset}
\end{equation}

with $H_\mathrm{cubic}$ being the cubic anisotropy field, $H_\mathrm{uniaxial}$ the uniaxial anisotropy field associated with the [100] direction, $\varphi_H$ the angle of the applied magnetic field, and $\varphi$ the magnetization direction within the film. An offset of the field $H_\mathrm{offset}$ is taken into account. Since the absolute orientation of the sample is unknown, an offset angle $\varphi_0$ is introduced. The data as well as the fit to the aforementioned function are depicted in Fig. \ref{fig:polar_plot}. 

\begin{table}[b]
    \centering
    \begin{tabular}{c  c  c  c  c}
        \toprule
        $H_\mathrm{cubic}$ & $H^{[100]}_\mathrm{uniaxial}$ & $H_\mathrm{offset}$ & $\varphi_0$ & $R^2$ \\
        \midrule
        \SI{77.7}{mT} & \SI{-7.4}{mT} & \SI{9.7}{mT} & -11.9° & 0.41\\
        \bottomrule
    \end{tabular}
    \caption{Fit results for the energy-based fit of the magnetic anisotropy.}
    \label{tab:polar_fit_results}
\end{table}

\subsection*{Fit Results and Discussion}
The resulting fit parameters and goodness of fit $R^2$ can be seen in tab. \ref{tab:polar_fit_results}. The negative result for the uniaxial anisotropy $H^{[100]}$ is physically questionable. In the same manner, the cubic term is way larger than what would be expected from the measured values. Considering the low goodness of the fit ($R^2 = 0.41$), these highly questionable results are logical. The discrepancy between model and data could stem from several factors and sources. First of all, the model might not be adequate to describe the specific material film at hand. Given the data which doesn't exhibit a clear symmetry or pattern, the model was chosen based mainly on flexibility rather than accurateness for the specific case at hand. Secondly, the comparatively low angular resolution (namely, 30° steps) leads to difficulties when fitting the model. Thirdly, the model seems to be insensitive to moderate variation of some of the parameters. This, in addition with the low number of points, causes highly unreliable fit results. Overall, more measurements points and the choice of a different, more adequate model could allow for more reliable fit results. 












